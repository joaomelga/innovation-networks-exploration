\documentclass[12pt]{article}
\usepackage[utf8]{inputenc}
\usepackage{amsmath}
\usepackage{amsfonts}
\usepackage{amssymb}
\usepackage{graphicx}
\usepackage{cite}
\usepackage{url}

\title{Venture Capital Network Structure and Nestedness Analysis}
\author{}
\date{}

\begin{document}

\maketitle

\section{Methodology}

\subsection{Data Source and Preprocessing}

This study uses data from Crunchbase, a broad database containing information about startups, venture capital firms, and investment rounds. The dataset includes information about companies, investors, investments, and funding rounds in the United States market. International venture capital firms from other countries also appear in the dataset when they participate in US startup investments.

The data preprocessing follows established methodologies from entrepreneurship literature \cite{Dalle2025}. The cleaning process implemented in our \texttt{data\_cleaning.py} module includes several steps: (1) removal of companies with incomplete information, (2) exclusion of companies founded after 2017 to allow sufficient time for investment patterns to emerge, (3) removal of companies with exit status (bankruptcy, acquisition, or IPO), and (4) application of a minimum funding threshold of \$150,000 to focus on substantive investment relationships.

To avoid endogeneity bias, companies that received funding only from accelerators were excluded from the analysis, following the approach described by Dalle et al. \cite{Dalle2025}. This ensures that observed network patterns reflect genuine investor relationships rather than program-specific effects.

\subsection{Investment Network Construction}

The analysis focuses on venture capital co-investment patterns across different funding stages. Investment stages are categorized into two main groups:
\begin{itemize}
    \item Early stages: angel, pre-seed, seed, and Series A
    \item Late stages: Series B through Series I
\end{itemize}

A bipartite network is constructed where nodes represent venture capital firms and edges represent co-investment relationships in the same company. The network is bipartite because it connects two distinct sets of investors: those participating in early-stage rounds (right nodes) and those participating in late-stage rounds (left nodes). This approach, implemented in our \texttt{network\_analysis.py} module, allows us to study how early-stage and late-stage investors interact in the investment ecosystem.

The bipartite graph $G = (U \cup V, E)$ consists of:
\begin{align}
U &= \{u_1, u_2, \ldots, u_m\} \text{ (late-stage VCs)} \\
V &= \{v_1, v_2, \ldots, v_n\} \text{ (early-stage VCs)} \\
E &\subseteq U \times V \text{ (co-investment relationships)}
\end{align}

To prevent spurious connections from related entities, investor pairs where the first five characters of their names match are filtered out, reducing the likelihood of including different funds from the same parent organization.

\subsection{Community Detection}

Community structure in the bipartite network is identified using the greedy modularity optimization algorithm \cite{Borgatti2011}. This method iteratively merges communities to maximize the modularity score, which measures the density of connections within communities compared to connections between communities.

For a bipartite network, modularity $Q$ is defined as:
\begin{equation}
Q = \frac{1}{2m} \sum_{i,j} \left[ A_{ij} - \frac{k_i k_j}{2m} \right] \delta(c_i, c_j)
\end{equation}

where $A_{ij}$ is the adjacency matrix, $k_i$ is the degree of node $i$, $m$ is the total number of edges, $c_i$ is the community of node $i$, and $\delta(c_i, c_j)$ is 1 if nodes $i$ and $j$ are in the same community, 0 otherwise.

The algorithm identifies communities of venture capital firms that frequently co-invest together, revealing structural patterns in the investment ecosystem that may not be apparent from individual investment decisions.

\subsection{Nestedness Analysis}

Nestedness is a structural property commonly observed in ecological networks \cite{AlmeidaNeto2008} that describes the tendency for specialists to interact with a subset of the partners of generalists. In the context of venture capital networks, nestedness would indicate that investors with fewer connections tend to co-invest with a subset of the partners of more connected investors.

We measure nestedness using the NODF (Nestedness based on Overlap and Decreasing Fill) metric \cite{AlmeidaNeto2008}, implemented in our \texttt{nestedness\_calculator.py} module. For a bipartite adjacency matrix $M$ with rows and columns sorted by decreasing degree, NODF is calculated as:

\begin{equation}
NODF = \frac{NODF_{rows} + NODF_{columns}}{2}
\end{equation}

where:
\begin{align}
NODF_{rows} &= \frac{100}{R(R-1)/2} \sum_{i=1}^{R-1} \sum_{j=i+1}^{R} \frac{|N_i \cap N_j|}{k_j} \text{ if } k_i > k_j \\
NODF_{columns} &= \frac{100}{C(C-1)/2} \sum_{i=1}^{C-1} \sum_{j=i+1}^{C} \frac{|N_i \cap N_j|}{k_j} \text{ if } k_i > k_j
\end{align}

Here, $R$ and $C$ are the number of rows and columns, $N_i$ represents the set of connections for node $i$, and $k_i$ is the degree of node $i$.

\subsection{Statistical Significance Testing}

To determine whether observed nestedness values are significantly higher than expected by chance, we employ a null model approach using the Curveball algorithm \cite{Strona2014}. This algorithm generates randomized matrices that preserve the degree sequence of both node sets while randomizing the connection patterns.

For each community, we generate 100 null matrices (@todo: generate 1000 instead) using 10,000 Curveball iterations. The statistical significance is assessed by comparing the observed NODF score against the distribution of null model scores:

\begin{equation}
Z = \frac{NODF_{observed} - \mu_{null}}{\sigma_{null}}
\end{equation}

where $\mu_{null}$ and $\sigma_{null}$ are the mean and standard deviation of the null distribution. Communities with $p < 0.05$ (where $p$ is the proportion of null models with NODF $\geq$ observed NODF) are considered to have significantly high nestedness.

\section{Results}

\subsection{Network Characteristics}

The analysis identifies a substantial venture capital co-investment network comprising [FIGURE 1: Network statistics to be inserted]. The bipartite network structure reveals clear separation between early-stage and late-stage investor communities, with varying degrees of connectivity between these groups.

[FIGURE 2: Network visualization showing bipartite structure]

Community detection using greedy modularity optimization identifies multiple investor communities of varying sizes. The algorithm discovers [NUMBER] distinct communities, with the largest communities containing over [NUMBER] investors each.

\subsection{Community Structure and Size Distribution}

The community size distribution follows a typical power-law pattern observed in many social networks. The top 10 communities by size account for the majority of investors in the network, suggesting a hierarchical organization within the venture capital ecosystem.

[FIGURE 3: Community size distribution]

Analysis focuses on communities with at least 100 nodes to ensure statistical power for nestedness analysis. This threshold excludes smaller communities that may not provide reliable nestedness measurements due to limited connectivity patterns.

\subsection{Nestedness Findings}

The nestedness analysis reveals significant heterogeneity across investor communities. Of the [NUMBER] communities analyzed, [NUMBER] show statistically significant nestedness (p < 0.05) compared to degree-preserving null models.

[FIGURE 4: Observed vs null nestedness scatter plot]

The distribution of Z-scores across communities indicates that most communities exhibit nestedness levels consistent with random networks, but several communities show substantially higher nestedness than expected by chance.

[FIGURE 5: Z-score distribution histogram]

Community [NUMBER] emerges as particularly interesting, showing the highest nestedness score (NODF = [VALUE]) with strong statistical significance (Z-score = [VALUE], p = [VALUE]). This community demonstrates a clear hierarchical structure where less connected investors tend to co-invest with a subset of the partners of more connected investors.

[FIGURE 6: Null model analysis for top nested community]

\subsection{Community Characterization}

Analysis of the geographic and sectoral characteristics of the most nested communities reveals distinct patterns:

\subsubsection{Geographic Distribution}

The highly nested communities show concentration in specific geographic regions, with [DESCRIPTION of geographic patterns].

[FIGURE 7: Geographic distribution across communities]

\subsubsection{Investment Stage Preferences}

Communities differ in their focus on specific investment stages. The most nested community shows [DESCRIPTION of stage preferences].

[FIGURE 8: Investment stage distribution]

\subsubsection{Sectoral Focus}

Sectoral analysis reveals that nested communities often specialize in particular technology sectors or industries.

[FIGURE 9: Sectoral distribution across communities]

\subsubsection{Funding Characteristics}

The funding patterns within nested communities differ from random networks, with [DESCRIPTION of funding patterns].

[FIGURE 10: Funding distribution analysis]

\subsection{Implications and Future Directions}

The discovery of significantly nested communities within the venture capital network has important implications for understanding investor behavior and startup access to capital. The hierarchical structure observed in these communities suggests the existence of informal investment hierarchies that may influence funding accessibility for entrepreneurs.

@todo: Investigate the economic consequences of nested community structure on startup success rates and funding efficiency.

@todo: Analyze the temporal evolution of community nestedness to understand how these structures emerge and persist over time.

@todo: Examine whether nested communities provide better or worse outcomes for portfolio companies compared to random investment patterns.

The presence of nested structures challenges the assumption of random mixing in venture capital markets and suggests that certain investors may serve as "gatekeepers" who influence access to broader investment networks. This finding aligns with social network theories about structural holes and brokerage positions \cite{Borgatti2011}.

@todo: Apply social network theories of structural holes to understand the role of highly connected investors in nested communities.

@todo: Investigate whether the nested structure reflects information asymmetries or risk-sharing mechanisms among investors.

The identification of these nested communities opens several avenues for future research into the social and economic mechanisms that drive venture capital ecosystem organization. Understanding these patterns may inform policy discussions about startup ecosystem development and investor network formation.

@todo: Develop theoretical models to explain the emergence of nested structures in investment networks.

@todo: Compare nestedness patterns across different geographic markets and time periods to understand generalizability.

This analysis provides the foundation for deeper investigation into how nested investor communities influence entrepreneurial ecosystems and capital allocation efficiency, which will be the focus of subsequent research phases.

\bibliographystyle{plain}
\bibliography{references}

\end{document}