\section{Introduction}

% -> Granoveter: To understand how people make economic decisions in the real world, you need to look at their social connections. We are neither isolated rational robots nor mere puppets of social norms. We are networks. And these networks shape our economic behaviors.

% - Talk about nestedness and bring concepts from "Nestedness in complex networks" for sure 

% - Talk about sindication

Two or more venture capital (VC) firms co-investing on the same enterprise is known in economics as syndication. In innovation networks, investors tend to behave like this so they can reduce the risk of investing in something not yet completelly validated or functional. In that case, reputation and centrality play an important role, once VCs will not only try to measure the potential return on investment (ROI), but also use its co-investors characteristics as a signal that interfers on their decisions.

The rising of syndicated investments in the last decades is evidence that innovation networks are by far a socialized network, where agents are not acting isolated, randomly, but in communities, being influenced by its peers. This phenomena is knwon as embededness.

Vast literature show how heavy tailled degree distributions emerge from such kind of interactions in social networks. In this same manner, innovation networks are not an exception, specially when it commes to number of connections of a certain player, as concentrated hubs of strongly connected agents can noramally be seen, while most part of the sample have only few ties, leading to a heterogenous distribtuion of connectance - also knwon as power-law degree distribution.

The widespread presence of power law degree distributions has incentivised numerous studies focused on uncovering plausible mechanisms behind their emergence, as well as exploring their impact on processes such as spreading dynamics \cite{PastorSatorras2001} and network robustness \cite{Albert2000}

When it comes to spreading dynamics, social and economic scientists have already explored how novel ideas are spread through networks, and how formation of bridge edges (with high betweeness) impact the chances of, for instance, novelty to spread in innovation networks. Well stablished ideas, like the Strengh of Weak Ties theory, are normally used as theoretical bases this kind of assumption.

In the other hand, measure the robustness to social and, being more specific, innnovation networks is still a theoretical and practical challenge. Impressivelly, ecology came to play an importat role to face it, and metrics like nestedness and the ecological consequences of its presence started to be transposed to social networks \cite{Theophile2024}.

On that paper, network theory is used to represent syndicated investments as edges of a network where investors are nodes with broad set of characteristics (geographic, financial, sectorial, etc.) This mathematical representation open horizions to better visualize and interpretate characteristics of this syndication network structure (or sub-networks inside of it) through ecology and economics lenses. 

Special attention is given for the fact that nestedness was observed among a certain group of early and late state investors.

