\section*{Abstract}

This study analyzes venture capital syndication networks using network theory to understand how investors organize and collaborate in funding startups. We examine 104,618 investment records from Crunchbase involving 38,843 investors and 16,932 companies to identify structural patterns in investor communities.

Using greedy modularity optimization, we discovered 170 distinct investor communities, with three large communities containing over 12,000 investors (75\% of the network). These communities exhibit power-law degree distributions typical of scale-free networks but differ significantly in investment activity and organization.

Our key finding is the emergence of significantly nested structures within Community 2, which is dominated by Silicon Valley investors. This hierarchical organization enables substantially higher transaction volumes (33.6\% of all investments) compared to similarly-sized communities. The nested structure suggests that informal investment hierarchies systematically influence funding accessibility for entrepreneurs.

% Temporal analysis reveals that nestedness emerged through a sharp phase transition in 2019 rather than gradual evolution. This transition occurred when network connectance reached approximately 0.026, suggesting threshold-dependent emergence mechanisms. The pattern shows three phases: non-significant period (2007-2018), rapid transition (2019), and sustained significance (2019-2024).

The concentration of nested structures specifically within Silicon Valley indicates that geographic clustering in innovation hubs may facilitate hierarchical investor relationships. High-degree investors in the nested community, particularly early-stage firms like SV Angel, function as network hubs that create efficient capital allocation pathways.

These findings challenge assumptions about random mixing in venture capital markets and suggest that network topology, rather than community size alone, determines investment ecosystem efficiency. The hierarchical organization may enhance stability against random investor departures but creates vulnerability to targeted removal of highly connected nodes.

Our results provide insights for ecosystem development strategies and suggest that understanding network structure is crucial for predicting investment patterns and improving access to capital for entrepreneurs.