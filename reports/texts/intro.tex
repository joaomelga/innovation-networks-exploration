\section{Introduction}

\todo[inline]{Improve introduction, it is still embryonic}

Venture capital syndication, where two or more firms co-invest in the same enterprise, represents a fundamental mechanism in innovation financing \cite{Granovetter1985}. This collaborative behavior emerges as investors seek to reduce risks associated with early-stage companies that lack complete validation or functional products. In such contexts, reputation and network centrality influence investment decisions, as venture capitalists use their co-investors' characteristics as signals that affect their own investment choices.

The increasing prevalence of syndicated investments in recent decades demonstrates that innovation networks are highly socialized systems where agents do not act in isolation but rather form communities influenced by their peers. This phenomenon reflects the concept of embeddedness, where economic actions are shaped by social structures and relationships \cite{Granovetter1985}.

Extensive literature documents how heavy-tailed degree distributions emerge from such interactions in social networks. Innovation networks follow similar patterns, particularly regarding investor connectivity, where concentrated hubs of highly connected agents coexist with a majority of participants having few connections. This heterogeneous distribution of connectivity follows a power-law degree distribution pattern.

The widespread presence of power-law degree distributions has motivated numerous studies investigating the mechanisms behind their emergence and their impact on processes such as spreading dynamics \cite{PastorSatorras2001} and network robustness \cite{Albert2000}.

Regarding spreading dynamics, social and economic scientists have explored how novel ideas propagate through networks and how bridge edges with high betweenness centrality affect the likelihood of innovation diffusion. Established theories such as the Strength of Weak Ties \cite{Granovetter1973} provide theoretical foundations for these analyses.

In the other hand, measuring robustness in social and innovation networks remains both a theoretical and practical challenge. Ecological network analysis has provided insights for addressing this challenge, with metrics such as nestedness and their ecological consequences being applied to social networks \cite{Theophile2024}.

This paper applies network theory to represent syndicated investments as edges in a network where investors serve as nodes with diverse characteristics (geographic, financial, sectoral). This mathematical representation enables visualization and interpretation of syndication network structures through both ecological and economic perspectives.

Particular attention is given to the observation of nestedness within a large community of early and late-stage investors in California. The implications and potential consequences of this phenomenon are explored in the \textit{Discussion and Implications} section of this paper.